\section*{\Huge DARMODĚJ}
\emph{Jaromír Nohavica}\\

\chord{Ami}Šel včera městem \chord{Emi}muž a šel po hlavní \chord{Ami}třídě,\chord{Emi}\\
\chord{Ami}šel včera městem \chord{Emi}muž a já ho z okna \chord{Ami}viděl,\chord{Emi}\\
\chord{C}na flétnu chorál \chord{G}hrál, znělo to jako \chord{Ami}zvon\\
a byl v tom všechen \chord{Emi}žal, ten krásný dlouhý \chord{F}tón,\\
a já jsem náhle \chord{F$\sharp$dim}věděl: Ano, to je \chord{E\textsuperscript{7}}on, to je \chord{Ami}on.\\

\begin{large}

Vyběh' jsem do ulic jen v noční košili,\\
v odpadcích z popelnic krysy se honily\\
a v teplých postelích lásky i nelásky\\
tiše se vrtěly rodinné obrázky,\\
a já chtěl odpověď na svoje otázky, otázky.\\

\end{large}

\textregistered: \chord{$2\times$: Ami, Emi, C, G, Ami, F, F$\sharp$dim, E\textsuperscript{7}} \emph{Na na na...}\\

\begin{large}

Dohnal jsem toho muže a chytl za kabát,\\
měl kabát z hadí kůže, šel z něho divný chlad,\\
a on se otočil, a oči plné vran,\\
a jizvy u očí, celý byl pobodán,\\
a já jsem náhle věděl kdo je onen pán onen pán\\

Celý se strachem chvěl, když jsem tak k němu došel,\\
a v ústech flétnu měl od Hieronyma Bosche,\\
stál měsíc nad domy jak čírka ve vodě,\\
jak moje svědomí, když zvrací v záchodě,\\
a já jsem náhle věděl: to je Darmoděj, můj Darmoděj.\\

\textregistered: \emph{Můj Darmoděj, vagabund osudů a lásek,\\
jenž prochází všemi sny, ale dnům vyhýbá se,\\
můj Darmoděj, krásné zlo, jed má pod jazykem,\\
když prodává po domech jehly se slovníkem.}\\

Šel včera městem muž, podomní obchodník,
šel, ale nejde už, krev skápla na chodník,\\
já jeho flétnu vzal a zněla jako zvon
a byl v tom všechen žal, ten krásný dlouhý tón,\\
a já jsem náhle věděl: ano, já jsem on, já jsem on. \hspace{1cm}
\textregistered

\end{large}

\newpage
